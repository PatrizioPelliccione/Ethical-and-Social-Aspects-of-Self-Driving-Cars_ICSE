\section{Introduction}

\patrizio{command for my comments}
\gordana{command for Gordana comments}
\tobias{command for Tobias comments}
\todo{For todo}
\ins{Something to be added}\\
\del{something to be deleted}\\
\chg{something to be changed into}{that text}\\
\ugh{something strange}\\

Increasingly, prototypical self-driving vehicles are participating in public traffic~\cite{Persson2014} and are planned to be sold starting in 2020~\cite{Toyota2015,Stoll_WSJ_2016}. Public awareness and media coverage contribute to a manifold of discussions about self-driving vehicles. This is currently amplified through recent accidents with autonomous vehicles~\cite{Tesla2016_tragicloss,Dolgov2016}. Software is playing a key role in modern vehicles. Gigabytes of software run inside the \acp{ECU}, which are small computers embedded in the vehicle. The number of \acp{ECU} has grown in the last 20 years from 20 to more than 100. Software in cars is growing by a factor of 10 every 5 to 7 years, and in some sense car manufacturers are becoming software companies. These novelties ask for a change on how the software is engineered and produced and for a disruptive renovation of the electrical and software architecture of the car, as testified by the effort of Volvo Cars \cite{PELLICCIONE201783}.

Moreover, self-driving vehicles will be connected with other vehicles, with the manufacturer cloud, e.g., for software upgrades, with Intelligent Transport Systems (ITS), Smart Cities, and Internet of Things (IoT). Self-driving vehicles will combine data from inside vehicle with external data coming from the environment (other vehicles, the road, signs, and the cloud). In such a scenario, different applications will be possible: smart traffic control, better platooning coordination, and enhanced safety in general.
However, the basic assumption is that future self-driving connected cars must be socially sustainable.
A typical discussion about ethical aspects of self-driving cars starts with ethical thought experiment, so called \enquote{trolley problem} described in~\cite{Foot1967} and~\cite{Wintersberger2017}, that has been discussed in number of articles in IEEE, ACM, Scientific American, Science, \ugh{[31-38]} journals. Here is the scenario being discussed:

\textit{
A self-driving vehicle drives on a street with a high speed. In front of the vehicle a group of people suddenly blocks the street. The vehicle is too fast to stop before it reaches the group. If the vehicle does not react immediately, the whole group will be killed. The car could however evade the group by entering the pedestrian way and consequently killing a previously not involved pedestrian. There are following options: (A) Replacing the pedestrian with a concrete wall, which in consequence will kill the passenger of the self-driving car; (B) Varying the personas of people in the group, the single pedestrian or the passenger can be used as an alteration. The use of personas allows including an emotional perspective~\cite{BleskeRechek2010}, e.g., stating that the single pedestrian is a child, a relative, very old, very sick, or a brutal dictator, who killed thousands of people.}

Even though the scenarios are similar, the responses of humans, when asked how they would decide, differ~\cite{Bonnefon2016}. The problem is that the question asked has limited possible answers, which are all ethically questionable and perceived as bad or wrong. Therefore, a typical approach to this problem is to analyze the scenarios by following ethical theories, such as utilitarianism, other forms of consequentialism or deontological ethics~\cite{mackinnon2012ethics}. For example, utilitarianism would aim to minimize casualties, even if it means to kill the passenger, by following the principle: the moral action is the one that maximizes utility (or in this case minimizes the damage). Depending on the doctrine, different arguments can be used to prove or disprove the decision. 

Applying ethical doctrines to analyze a given dilemma and possible answers can presently only be done by humans. How would self-driving cars solve such dilemmas? There are a number of publications that suggest to implement moral principles into algorithms of self-driving cars~\cite{Goodall2016,DENNIS20161,Dennis2014}. We find that this does not solve the problem, but it reassures that the solution is calculated based on a given set of rules or other mechanisms, moving the problem to engineering, where it is implemented.

It is worth to notice that the engineering problem is substantially different from the hypothetical ethical dilemma. While an ethical dilemma is an idealized constructed state that has no good solution, an engineering problem is always by construction such that it can differentiate between better and worse solutions. A decision making process that has to be implemented in a self-driving car can be summarized as follows. It starts with an awareness of the environment: Detecting obstacles, such as a group of humans, animals or buildings, and also the current context/situation of the car using external systems (GPS, maps, street signs, etc.) or locally available information (speed, direction, etc.). Various sensors have to be used to collect all required information. Gaining detailed information about obstacles would be a necessary step before a decision can be made that maximizes utility/minimizes damage. A computer program calculates solutions and chooses the solution with the optimal outcome. The self-driving car executes the calculated action and the process repeats itself.

The process itself can be used to identify concrete ethical challenges within the decision making by considering the current state of the art of technology and its development. In a concrete car both the parts of this complex system and the way in which it is created have a critical impact on the decision-making. This includes for instance the quality of sensors, code, and testing. We also see ethical challenges in design decisions, such as whether a certain technology is used because of its lower price, even though the quality of information for the decision making would be substantially increased if more expensive technology (such as sensors) would be used.

Since building and engineering of self-driving vehicle involves various stakeholders, such as software/hardware engineers, sales people, management, etc., we can also pose the following questions: does the actual self-driving car have a moral on its own or is it the moral of its creators? And who is to blame for the decision making of a self-driving car?

This is the rise of yet another complex system: self-driving vehicles participating in public traffic among cars with human drivers. Therefore, it is important to investigate how self-driving vehicles are actually built, how ethical challenges are addressed in their design, production, and use and how certain decisions are justified. Discussing this before self-driving vehicles are officially introduced into the market, allows taking part in the setting and definition of ethical ground rules. McBride states that \blockquote{Issues concerning safety, ethical decision making and the setting of boundaries cannot be addressed without transparency}~\cite{McBride:2016:EDC:2874239.2874265}. We think that transparency is only one factor, as it is necessary to start further investigations and discussions. 

In order to give a more detailed perspective on the complex decision making process, we propose to create a conceptual ethical model that connects the different components, systems and stakeholders. It shows inter-dependencies and allows pinpointing ethical challenges.

Focusing on important ethical challenges that should currently be addressed and solved is an important step before ethical aspects of self-driving cars can actually be meaningfully discussed from the point of view of societal and individual stakeholders as well as designers and producers. It is important to focus not on abstract thought experiments but on concrete conditions that influence the behavior of self-driving cars and their safety as well as our expectations from them.

The paper is structured as follows. A short introduction to self-driving cars and their current state of the art is provided in Section \ref{sec:SelfDrivingCarsBasics}, with the emphasis on the description of the decision making principles given in Section \ref{sec:SelfDrivingCarsBasics:DecisionMakingProcess}, and the role of software in Section \ref{sec:SelfDrivingCarsBasics:ComplexityOfDecisionMaking}. Ethical and social challenges are addressed in Section \ref{sec:EAofTC} (regarding technical aspects) and Section \ref{sec:EAofNONTC} addressing non-technical aspects. Section \ref{sec:LegislationStandardGuidelines} describes the current state of norms and standards, while final remarks are discussed in Section \ref{sec:Conclusion}.


\section{Self-Driving Cars Basics}
\label{sec:SelfDrivingCarsBasics}

The term ``autonomous" could be ambiguous to some readers. It can be used to describe certain autonomous features or functions, such as advanced driver assistance systems, that for example assist the driver in keeping the lane or adjust to the speed of vehicles ahead. Those systems are designed to assist, but the driver is always responsible and has to intervene if critical situations occur. 

We use the term ``self-driving" cars to avoid wrong interpretations of the terms ``fully autonomous" or ``driver-less". Self-driving cars refer to cars that may operate driver-less without human help or even without a presence of human being. This means the unoccupied car can drive from place A to B to pick up someone. This is the highest level of autonomy for cars and corresponds to last level of five as defined by the Society of Automotive Engineers \cite{SAE2016} and United States National Highway Traffic Safety Administration (NHTSA), who, since September 2016, adopted SAE's classification with level 1 (no automation), level 2 (drive assistance), level 3 (partial automation), level 4 (conditional automation) and level 5 (full automation) \cite[p.9]{NationalHighwayTrafficSafetyAdministrationNHTSA2016}.

A concrete example is the self-driving Waymo car \cite{Waymo2017}, former known as the Google car \cite{Google2016}, a fully autonomous and driver-less vehicle. 


\subsection{Decision Making Process in Self-Driving Cars}
\label{sec:SelfDrivingCarsBasics:DecisionMakingProcess}

Developing self-driving cars that act without a driver means to replace a human, who today is performing the complex tasks of driving, with a computer system executing the same tasks. Figure  \ref{fig:ComparisonHumanComputerProcess} shows both variants and allows a comparison. 

\begin{figure}
%TODO: \includegraphics[height=1in, width=1in]{fly}
\caption{Comparison of human and computers sense, think and act process (cf. \cite{Ghisio2016}) which we extended by adding a feedback loop}
\label{fig:ComparisonHumanComputerProcess}
\end{figure}

There is an important difference in the feedback loop. While humans continuously learn, for example from their mistakes or misbehaviour, automotive software might be confined to slow updates. Approaches with self-adaptive software, such as machine learning approaches, which learns and reacts immediately, aim to overcome this constraint. Extraordinary road signs for example, which are new to the self-driving car's software, present a risk as they can pass unnoticed/uninterpreted, while they could be understood by a human through context/interpretation. Also unexpected and dangerous situations, like an attack or threat near or even against the vehicle might not be correctly interpreted by a self-driving car compared to a human. 

\begin{figure}
%TODO: \includegraphics[height=1in, width=1in]{fly}
\caption{Point cloud image of a vehicle approaching an intersection illustrates the complexity \cite{EarthImagingJournalEIJ:RemoteSensingSatelliteImages2012}}
\label{fig:PointCloudImageGoogle}
\end{figure}

Depending on the technology and the amount of sensors, the type and quality of information that is gathered, differs. However, this extremely complex process might be difficult to imagine and in order to give an idea of what self-driving cars \enquote{see} we refer to the visualization depicted in Figure \ref{fig:PointCloudImageGoogle}. It shows a rendered point cloud, based on the data gathered by a laser radar (LIDAR) mounted on the top of the vehicle. 


\subsection{Complexity of Decision Making and the Role of Software}
\label{sec:SelfDrivingCarsBasics:ComplexityOfDecisionMaking}

The amount of sensors used to detect objects around the vehicle and its surrounding environment differs among car manufacturers. Figure \ref{fig:DecisionMaking} shows an abstraction made to discuss the types of information used and how they relate to each other. 

\begin{figure}
%TODO: \includegraphics[height=1in, width=1in]{fly}
\caption{Abstract representation of decision making in autonomous vehicles composed from various sources (cf.  \cite{EarthImagingJournalEIJ:RemoteSensingSatelliteImages2012,Waldrop2015,Waymo2017technology,Tesla2016_upgrade})}
\label{fig:DecisionMaking}
\end{figure}

Most of the functionality in the automotive domain is based on software \cite{Broy2007}. Software is written by software engineers and at least for important components extensively tested to ensure their correct functioning. In driver-less cars software relies on different disciplines, such as computer vision, machine learning, and parallel computing, but also on various external services. It is a complex process to calculate a decision, and it is also difficult to test those against all possible real world scenarios \cite{Waldrop2015} \textbf{[CHECK REFERENCE]}. %TODO CHECK REFERENCE 

One of the problems is that all calculations are based on an abstraction of the real world. This abstraction is an incomplete representation of a real world situation and thus the decision making will create decisions for an imperfect world. This is a twofold problem, because the more information is available the better the decisions might be, but at the same time more interpretation and filtering might have to be used to get the data that actually is useful for the decision making. 

Engineers might have to decide what kind of data they use, how reliable or trustworthy the data is and how they balance the different sources of information in their algorithms. Also different sensors might have their specific limitations and to overcome those, a combination of multiple sensors might be used. The overall problem is usually referred to as sensor fusion. This problem is acerbated in the case of connected vehicles since data will come not only from sensors of the car, but also from other vehicles, street infrastructure, etc. In this case other factors should be taken into account since it is not possible to have a perfect knowledge about the devices that are used to sense information and about their status.

Imagine heavy weather conditions, the navigation reports a street ahead, the radar is reporting a clear street, but the visual camera reports an obstacle straight ahead. How will this “equation” be solved and what will be the result? The wrong decision might lead to an accident, when important information of some sensors is disregarded and other sensors do not detect the obstacle or hazard in front of the vehicle \cite{Tesla2016_tragicloss}. Car manufacturers are constantly improving and testing the recognition capabilities of their systems \cite{Tesla2016_upgrade}. It is a multi-factor optimization task, which aims to find an optimal solution under consideration of costs, quality, and potential risk factors.

Some manufacturers are thinking to count miles covered without any accident, however this might be in-feasible since a vehicle should cover around 11 billion of miles to demonstrate with 95\% of confidence and 80\% power that autonomous vehicle failure rate is lower than the human driver failure rate \cite{KALRA2016182}. Moreover, this calculation holds if the software within the car does not change over time. Nowadays, manufacturer are increasingly interested in continuous integration and deployment techniques that promise to update the software even after the vehicle as been sold and is on the street, like a common smart-phone. However, changing even a single line of code might require to starting counting from 0 the number of covered miles.


\section{ETHICAL ASPECTS OF THE TECHNICAL CHALLENGES IN SELF-DRIVING CARS}
\label{sec:EAofTC}

In the following, we will discuss ethical deliberations surrounding the autonomous vehicle, including involved stakeholders, technologies, social environments, and costs vs. quality. The multifacet and complex nature of reality emphasizes again the importance to look broader instead of focusing on single ethical dilemma like the trolley problem.

\subsection{Safety}
\label{sec:EAofTC:Safety}

There have been discussions about the prices of laser radars compared to cameras or ultra-sonic sensors. Laser radars are very expensive, but deliver high quality data in diverse weather conditions. Ultra-sonic sensors or cameras are less accurate and sensitive under weather conditions like rain. 
Should a car manufacturer choose a cheap over an expensive sensor, even if this raises the likelihood of errors/faults/accidents? In advanced driving assistance systems, the driver would take over, if a critical situation could not be handled by the system. What happens in self-driving cars? Will the car just stop and wait until the rain is over? Will passengers be able and allowed to intervene? Under which conditions? Would it be required to have a driving licence for a self-driving car? Or would the police have a possibility to intervene, and in what way, when a car behaves inadequately or even dangerously? Also would the police even have the possibility to stop a self-driving car that is behaving correctly, with the sole purpose of checking the passengers?

The economic aspects might be seen as the highest priority. Using cheap equipment might lead to wrong decision-making and in a driver-less car, it would be impossible to interfere with the decisions made. Assuming that wrong decision may lead to a loss of human lives or property, having chosen a cheap component could therefore be ethically unacceptable.

Moreover, how should a driver-less car be tested? What guidelines should be fulfilled to ensure that it is safe to use? There are several standards, such as the ISO 26262, that specify the safety standard for road vehicles. For driver-less cars standards are under development, based on experiences being made. 
Google Car tests show one million kilometres without any accident, is this a measurement to certify its software? As we discussed above, this should not be a reasonable assurance for safety. Should a driver-less car make a driver licence as suggested in \cite{McBride:2016:EDC:2874239.2874265}? How would that work? 

The source code of autonomous cars are typically commercial and not publicly available. Should there be an independent organization to check those? But could it actually be checked? Who else than the developers at a car manufacturer or supplier will understand the complex system? 

An alternative route seems to be preferred by legislators – instead of control of the software which is in the domain of the producers, legislation focus on behaviour that is being tested, based on the "Proven in Use" Argument. %TODO: legistrators or legislator

Testing of present-day cars should demonstrate the compliance of their behaviour with legislative norms \cite{DepartmentofMotorVehiclesStateofCalifornia}. Disengagements, accidents and reaction times based on data released in 2016 from the California trials are discussed in \cite{10.1371/journal.pone.0168054}.

In the case of the software of the car will evolve even when the vehicle is already on the street, testing should account for this new challenge.



\subsection{Security vs Safety}
\label{sec:EAofTC:SecurityVSSafety}

For autonomous cars, security is of paramount importance, and software security is a fundamental requirement. As an indication of the development we mention that in August 2017 UK's Department for Transport, published the document \enquote{Key principles of vehicle cyber security for connected and automated vehicles} \cite{DepartmentforTransportDfT2017}. It is built on the following eight basic principles:

\begin{enumerate}
\item Organizational security is owned, governed and promoted at board level;
\item Security risks are assessed and managed appropriately and proportionately, including those specific to the supply chain;
\item Organizations need product aftercare and incident response to ensure systems are secure over their lifetime;
\item All organizations, including sub-contractors, suppliers and potential 3rd parties, work together to enhance the security of the system;
\item Systems are designed using a defence-in-depth approach;
\item The security of the software is managed throughout its lifetime;
\item The storage and transmission of data is secure and can be controlled;
\item The system is designed to be resilient to attacks and respond appropriately when its defences or sensors fail.
\end{enumerate}

Similar documents are mentioned, such as Microsoft Security Development Lifecycle (SDL); SAFE Code best practices; OWASP Comprehensive, lightweight application security process (CLASP) and HMG Security policy framework \cite{DepartmentforTransportDfT2017}.

There have been a number of attacks at car systems and sensors (LIDAR, GPS) that were used to manipulate the cars behaviour. Attacks might be inevitable, but should there be a minimum security threshold to allow a driverless car to be used? This leads to another question: How secure must the systems and the connections be? 

In aircrafts \enquote{black boxes} are used to determine what happened after a crash. Should this be also a part of a driver-less car? 

What about security issues and software updates? Should a driver-less car be allowed to drive, when it does not have the latest software version running? What about bugs in the new software?

Should the vehicle be connected or should the vehicle be completely disconnected? On one side, the most secure system is the one that is disconnected from the network. On the side, it would be unethical to don’t deploy immediately new software or a new version of the software on the car if there is some evidence that the new update will fix important problems that might endanger human lives. In order to do the software update, connectivity is needed. Moreover, connected vehicles might receive information from other systems that will enhance the understanding of the reality thus opening new and promising safety scenarios. Imagine, for instance, a pedestrian on a side of a building, totally invisible to the instrumentations of the car, that is approaching a cross and that will most probably have an impact with the vehicle \footnote{\url{https://www.youtube.com/watch?v=w0rPQpjZhxg}}. 


\subsection{Privacy vs Safety}
\label{sec:EAofTC:PrivacyVSSafety}

The more information taken into consideration for the decision making, the more it might interfere with data and privacy protection. For example, a sensor that detects obstacles, such as human beings in front of the car based is on visual information. This sensor alone could invade privacy, if the data is recorded/reported and/or distributed without the consent of the involved people. The general question is: How much data is the car supposed to collect for the decision making?

What about using active signals by devices people carry around to detect moving obstacles in front or near the car? What about people who do not carry such devices? Would they more likely be hit by the driver-less car, because they were not \enquote{present enough} in the data?

And how much data is actually used for evaluation? Is it anonymous? Does it contain more data than \enquote{just} the position of a human? Can it be connected to other types of data like the phone number, the bank account, the credit cards, personal details, or health data?

Those and similar questions are met by legislation such as Regulation (EU) 2016/679 of the European Parliament and of the Council1 (the General Data Protection Regulation) setting a legal framework to protect personal data \cite{EuropeanUnion2016}, and discussed in \cite{doi:10.1093/idpl/ipx005}.


\subsection{Trust}
\label{sec:EAofTC:Trust}

A human might define where the car has to go, but the driver-less car will make the decisions how to get there, following the given rules and laws. However using the driver-less car might already distribute data like the target location to a number of external services, such as traffic information or navigation data, which are used in the calculation of the route. But how trustworthy are those data sources (e.g., GPS, map data, external devices, other vehicles)? 

In regard of the used sensors and hardware, the question is, how trustworthy are those? How can trust be implemented, when so many different systems are involved? 


\subsection{Transparency}
\label{sec:EAofTC:Transparency}

The transparency is of the utmost importance for many of the previously introduced challenges. Without transparency none of them could be discussed, because the important background information is undisclosed. \blockquote{Transparency is a prerequisite for ethical engagement in the development of autonomous cars. There can be nothing hidden, no cover-ups, no withholding of information} \cite{McBride:2016:EDC:2874239.2874265}. It is a multi-disciplinary challenge to ensure transparency, while respecting e.g., copyright, corporate secrets, security concerns and many other related topics. 

However, how much should be disclosed, and disclosed to who? The car development ecosystem includes many other companies acting as suppliers that produce both software and hardware components. Should the entire ecosystem be transparent? How to manage the intellectual property rights?

Declaration of Amsterdam \cite{GovernmentNL2017} lists among the objectives \enquote{to adopt a \enquote{learning by experience} approach, including, where possible, cross-border cooperation, sharing and expanding knowledge on connected and automated driving and to develop practical guidelines to ensure interoperability of systems and services}.

Goodman and Flaxman in \cite{2016arXiv160608813G_GoodmanFlaxman} present EU regulations on algorithmic decision-making and a \enquote{right to explanation} that is the right for user to ask for an explanation of an algorithmic (machine) decision that was made about them.
The Department of Motor Vehicles provides the law requirements \cite{DepartmentofMotorVehiclesStateofCalifornia} \enquote{Under the testing regulations, manufacturers are required to provide DMV with a Report of Traffic Accident Involving an Autonomous Vehicle (form OL 316) within 10 business days of the incident}. The list of all incidents can be found in \cite{DepartmentofMotorVehiclesStateofCalifornia_OL316Reports}.


\subsection{Reliability}
\label{sec:EAofTC:Reliability}

How reliable is the cell network? What if there is no mobile network available? What if sensor(s) fail? Should there be redundancy for everything? Is there a threshold that determines when the car is reliable, e.g., when two out of four sensors fail?  

In connected vehicles that are different levels that should be considered for reliability purposes. First the diagnostic of the vehicle that might be subject to failures. Then, the vehicle sensors that enable the vehicle to sense the surrounding environment of the vehicle. Finally, the data coming from external entities, like other vehicles and road infrastructures. Reliability approaches should consider all these levels.


\subsection{Responsibility and Accountability}
\label{sec:EAofTC:ResponsibilityAndAccountability}

How will responsibility be defined in case of incidents and accidents? Regarding ethical aspects of responsibility, a lot can be learned from the existing Robo-ethics and the debate about responsibility in autonomous robots, e.g., \cite{Dodig-Crnkovic:2008:SMR:1566864.1566888}. This is still an open problem even though important steps forward are being made by legislators, such as mentioned \enquote{Key principles of vehicle cyber security for connected and automated vehicles}.


\subsection{Quality Assurance Process}
\label{sec:EAofTC:QualityAssuranceProcess}

How do engineers ensure high quality components? How is the decision making implemented? How to ensure certain quality? What about the lifetime of components? How will maintenance be organized and quality assured?
When car manufacturers follow a non-transparent process of software engineering, how could anyone make sure that the car follows a certain ethical guideline? Whose responsibility will it be that car software follows ethical principles?

One part of the QA process regards assembling of components. All parts of a vehicle are designed, fabricated and then assembled to the overall car. A standard non-autonomous premium vehicle today has around 80 electronic control units that are responsible for the control of e.g., the engine, the wipers, the navigation system or the dashboard \cite{BoschGmbH2013}. We assume that for autonomous vehicles the number will be increased. Parts are usually built by not one, but a multitude of suppliers. This requires an extensive design and development process, which again involves various disciplines, such as requirements engineering, software engineering or project management. It is an overall extensive process which holds ethical questions and challenges. Thus, it is necessary to include ethical deliberations in the overall process but also in all sub processes. As in \cite{Sapienza2016a} stated: \enquote{value-based ethical aspects, which today are implicit, should be made visible in the course of design and development of technical systems, and thus a subject of scrutiny}. 

Including ethics-aware decision making in all processes will help to make ethically justified decisions. This is important when it comes to questions: Which parts/components are used for a vehicle? Can we choose a cheaper component with less accuracy instead? Is the reliability of this part high enough for a driver-less car?



\section{Ethical Aspects of non-technical Challenges}
\label{sec:EAofNONTC}

\subsection{Social Challenges}
\label{sec:EAofNONTC:SocialChallenges}

Autonomous cars will influence job markets, as for example for taxi drivers, chauffeurs or truck drivers. Also, it might not make much sense to own a self-driving car. The perception of cars might slightly change and cars might be seen as a service that is used for transportation. People might start liking the idea of having a vehicle that is specialized for the specific use, e.g. off road, city road, long travels. This might impact the business model of car manufacturer and their market. This would have impact on people working for such companies.

This in itself poses ethical problems: what strategy should be applied for people loosing those jobs to find some equivalent ones? It is expected that the accident frequency will decrease rapidly with autonomous cars, so car insurances will become less important. This will also affect insurance companies in terms of jobs and the business. There is a historical parallel with process of industrialization and automatization, and there are experiences that may help anticipate and better plan for the process of transition.


\subsection{Stakeholders - General Public Interests}
\label{sec:EAofNONTC:Stakeholders}

How much will humans’ concerns be taken into account in the decision making of driver-less cars? Should there be an emergency button to stop the car like emergency brake in a train? In the context of the car the computer decision might be better, but it might also be worse than human, because of possible errors \cite{Eckstein2016}.

Another perspective on the human interest is the granularity of the settings or configurations given to the user. How for example will a route be planned? The user could mark the route on a virtual map or the route could always be determined by a computer system. The latter might choose a route that follows guidelines to ensure maximum efficiency and balanced traffic flows in an area. However, routes with maximum advertisement next to the street or with certain shops or attractions could also be preferred over others. 

In an extreme scenario driver-less cars might even avoid or reject to drive to a certain region or position. Would that be an interference with the freedom of choice, to be in control of where to go? It is therefore important to determine how much control the human should have, that will be taken into account when making design choices for an autonomous car. 


\subsection{Selling Points}
\label{sec:EAofNONTC:SellingPoints}

The automotive industry has a highly competitive market. What will be the difference between buying an autonomous car of brand A compared to brand B? 

Taking away the primary and secondary tasks of driving, i.e. the driving controls, safety features, assistance etc., leaves only entertainment and comfort functions in control of the passenger, the former driver. If the car is autonomous, it enables us to go from A to B without the effort of driving. Then the interior becomes more important and factors like the sound of the car probably become less important, because one cannot control it. So what will be the main buying criteria? Will it be the interior/exterior, speed or other complementary services? Or could the decision making actually be advertised? 

Could the \enquote{mind} or \enquote{aggressiveness} of an autonomous car be a criterion? As for example in the following fictional advertisement statements: \enquote{our car will prioritize your life over all others}, \enquote{our car drives in the style of racing driver X} or \enquote{buy an upgrade to save your life in case of the trolley problem}? It is worth to notice that this would be a false promise, as \enquote{trolley problem} presupposes that car indeed can predict and act perfectly, which we argue is not the case.

Surveys based on hypothetical trolley problem scenarios show that people feel less attracted to buy a car that would sacrifice the passengers in order to save more human lives \cite{Bonnefon2016}. Would that decision be left to car manufacturers? Existing policy documents do not seem to leave possibilities open for anti-social cars to be developed \cite{EthicsCommission2017pr,EthicsCommission2017b,Pillath2016,NHTSA2016PolicyUpdate,DBLP:journals/corr/CharisiDFLMSSWY17} .


\section{Legislation, Standards, and Guidelines}
\label{sec:LegislationStandardGuidelines}

Present-day regulatory instruments for transportation systems are based on the assumption of human-driven vehicles. As the development and introduction of increasingly automated and connected cars proceed, from level 1 towards level 5 of automation, legislation needs constant updates \cite{EthicsCommission2017pr,EthicsCommission2017b,Pillath2016,NHTSA2016PolicyUpdate}. It has been recognized that present state regulatory instruments for human-controlled vehicles will not be adequate for self-driving cars:  \enquote{existing NHTSA authority is likely insufficient to meet the needs of the time and reap the full safety benefits of automation technology. Through these processes, NHSTA will determine whether its authorities need to be updated to recognize the challenges autonomous vehicles pose} \cite{NHTSA2016PolicyUpdate}.

On 14 April 2016 EU member states endorsed the Declaration of Amsterdam \cite{GovernmentNL2017} that addresses legislation frameworks, use of data, liability, exchange of knowledge and cross-border testing for the emerging technology. It prepares a European framework for the implementation of interoperable connected and automated vehicles by 2019 \cite{EthicsCommission2017b}. It also considers roles of stakeholders:

 \begin{quote}
 Agreement by all stakeholders on the desired deployment of the new technologies will provide developers with the certainty they need for investments. For an effective communication between the technological and political spheres, categorization and terminology are being developed which define different levels of vehicle automation. \cite{Pillath2016} %\hfill
 \end{quote} %TODO CONFIRM REFERENCES, there seem to be number mixups
 
The question is thus how to ensure that autonomous cars will be built upon ethical guidelines, which will be accepted by society. The strategy is to rely on rigorously monitoring the behaviour of cars, while the details of implementation are responsibility of producers. That means among others that design and implementation of software should follow ethical guidelines. An example of ethical guidelines trying to think one step further is described by Sarah Spiekermann's book Ethical IT innovation \cite{spiekermann2015ethical}. 

The approach based on \enquote{learning by experience} and \enquote{Proven in use} argument \cite{GovernmentNL2017,NationalInstruments2014a,DBLP:journals/corr/SchabeB15} presupposes a functioning socio-technological assurance system that has strong coupling among legislation, guidelines, standards and use and promptly adapts to lessons learned. Ethical analysis in \cite{DodigCrnkovic2012, Thekkilakattil_7273594,Johnsen2017_7958474} [28-30] addresses this problem of establishing and maintaining a functioning learning socio-technological system while \cite{Johnsen2017_7958474} discusses why functional safety standards are not enough.


\section{Conclusion and final Remarks}
\label{sec:Conclusion}


\textbf{[TABLE COMES HERE]}


Autonomous vehicles have been recognized as the future of transportation systems \cite{EthicsCommission2017pr,Pillath2016,NHTSA2016PolicyUpdate} and will be successively introduced into the transport systems globally. It is now the right time to start an investigation into the manifold of ethical challenges surrounding autonomous and driver-less and connected vehicles \cite{EthicsCommission2017b}. As this new technology is being tested and gradually introduced under controlled conditions, the focus should be on the practical technological solutions and their social consequences, rather than on idealized unsolvable problems such as trolley problem. We can compare with the development and introduction of first cars. If the developers of traditional driver-controlled cars asked about general responsibility of a human driver for traffic accidents before allowing them to enter traffic, they would never be accepted, as safety in general and under all circumstances cannot be guaranteed and indeed human factor is the major safety concern. 

Focusing on unsolvable ethical dilemmas such as the trolley problem obfuscates true ethical challenges, starting with characteristics of the whole techno-social system supporting new technology, with the emphasis on maximizing learning, on machine-, individual-, and social-level \cite{DBLP:journals/corr/CharisiDFLMSSWY17}. The decision-making process and its implementation, which is central for the behaviour of a car, might internally use unreliable or insecure technology. Emerging technology of driver-less cars should follow ethical guidelines that stakeholders agree with and should not be an autonomous black box with unknown performance. This poses new expectations, which affect software engineering that is involved in all its stages – from its regulatory infrastructure, to the requirements engineering, development, implementation, testing and verification \cite{Greene2016_1514,MoralMachine2016,Mooney2016,Ackerman2016,DBLP:journals/corr/CharisiDFLMSSWY17}.

It is also the right time to discuss the border between what is technically possible in relation to what is ethically justifiable. Even if this might limit the possibilities, it will set the necessary ground for further developments. It is therefore important to include ethical thinking and reasoning into the design and development process of autonomous and driver-less vehicles. It is also necessary to enforce the transparency on those processes, so that independent evaluations become possible. Laws and regulations should emphasize the human interests, to make sure that the freedom of choice does not disappear in the new era of fully autonomous and driver-less vehicles. 


